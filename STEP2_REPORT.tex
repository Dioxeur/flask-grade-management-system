\documentclass[12pt,a4paper]{article}
\usepackage[utf8]{inputenc}
\usepackage[english]{babel}
\usepackage{geometry}
\usepackage{fancyhdr}
\usepackage{graphicx}
\usepackage{hyperref}
\usepackage{listings}
\usepackage{xcolor}
\usepackage{booktabs}
\usepackage{longtable}
\usepackage{array}
\usepackage{tikz}
\usetikzlibrary{shapes.geometric, arrows, positioning}

% Page setup
\geometry{margin=1in}
\pagestyle{fancy}
\fancyhf{}
\rhead{Step 2 Report}
\lhead{Flask Grade Management System}
\cfoot{\thepage}

% Code listing setup
\lstset{
    basicstyle=\ttfamily\small,
    backgroundcolor=\color{gray!10},
    frame=single,
    breaklines=true,
    showstringspaces=false
}

% Hyperlink setup
\hypersetup{
    colorlinks=true,
    linkcolor=blue,
    urlcolor=blue,
    citecolor=blue
}

\title{\textbf{Step 2 Report: Database \& Website Prototype} \\ 
       \large Database Design, Web Functionalities \& System Architecture \\
       \vspace{0.5cm}
       \large Flask Grade Management System}
\author{Gabriel Mokhele (Dioxeur)}
\date{\today}

\begin{document}

\maketitle
\thispagestyle{empty}

\newpage

\tableofcontents
\newpage

\section{Executive Summary}

This report documents the successful completion of Step 2 deliverables for the Flask Grade Management System project. The primary objectives included implementing a functional database system, developing core web functionalities, and creating a working prototype with operational features.

\subsection{Project Overview}
\begin{itemize}
    \item \textbf{Project Name:} Flask Grade Management System
    \item \textbf{Developer:} Gabriel Mokhele (GitHub: Dioxeur)
    \item \textbf{Repository:} \url{https://github.com/Dioxeur/flask-grade-management-system}
    \item \textbf{Status:} \textcolor{green}{\textbf{COMPLETED}}
    \item \textbf{Completion Date:} \today
\end{itemize}

\section{Deliverables Status}

\begin{longtable}{|p{2cm}|p{4cm}|p{2cm}|p{6cm}|}
\hline
\textbf{Step} & \textbf{Deliverable} & \textbf{Due Date} & \textbf{Format} \\
\hline
\endhead
2 & Prototype with database & Day 2 & Database set, web site functionalities are operational \\
\hline
\end{longtable}

\section{Database Architecture}

\subsection{MySQL Database Implementation}

The system utilizes a MySQL database named \texttt{user} hosted locally via XAMPP. The database consists of four interconnected tables that manage the complete grade management workflow.

\subsubsection{Database Connection Configuration}
\begin{lstlisting}[caption=Database Configuration]
SQLALCHEMY_DATABASE_URI = 'mysql+mysqlconnector://root:@localhost/user'
SQLALCHEMY_TRACK_MODIFICATIONS = False
\end{lstlisting}

\subsection{Entity Relationship Diagram (ERD)}

\begin{figure}[h]
\centering
\includegraphics[width=0.8\textwidth]{erd_screenshot.png}
\caption{Entity Relationship Diagram - Database Structure}
\label{fig:erd}
\end{figure}

The ERD demonstrates the complete database architecture with the following key relationships:

\begin{itemize}
    \item \textbf{USER} entity serves as the central hub for both students and teachers
    \item \textbf{SUBJECT} entity connects to USER through teacher\_id (one teacher per subject)
    \item \textbf{GRADE} entity maintains three foreign key relationships:
    \begin{itemize}
        \item student\_id → USER (which student received the grade)
        \item subject\_id → SUBJECT (which subject the grade belongs to)
        \item teacher\_id → USER (which teacher assigned the grade)
    \end{itemize}
    \item \textbf{NOTE} entity connects to USER through user\_id for personal notes
\end{itemize}

\subsection{Database Relationship Analysis}

From the ERD, we can identify the following relationship cardinalities:

\begin{itemize}
    \item \textbf{One-to-Many:} USER (Teacher) → SUBJECT (A teacher can teach multiple subjects)
    \item \textbf{Many-to-One:} GRADE → USER (Student) (A student can have multiple grades)
    \item \textbf{Many-to-One:} GRADE → USER (Teacher) (A teacher can assign multiple grades)
    \item \textbf{Many-to-One:} GRADE → SUBJECT (A subject can have multiple grades)
    \item \textbf{One-to-Many:} USER → NOTE (A user can write multiple notes)
\end{itemize}

\subsection{Table Structures}

\subsubsection{USER Table}
\begin{longtable}{|p{3cm}|p{2cm}|p{2cm}|p{6cm}|}
\hline
\textbf{Column} & \textbf{Type} & \textbf{Constraints} & \textbf{Description} \\
\hline
\endhead
id & int(11) & PRIMARY KEY, AUTO\_INCREMENT & Unique user identifier \\
\hline
email & varchar(150) & UNIQUE, NOT NULL & User email address \\
\hline
password & varchar(150) & NOT NULL & Hashed password \\
\hline
first\_name & varchar(150) & NOT NULL & User's first name \\
\hline
role & varchar(20) & NOT NULL & User role: 'student' or 'teacher' \\
\hline
\end{longtable}

\subsubsection{SUBJECT Table}
\begin{longtable}{|p{3cm}|p{2cm}|p{2cm}|p{6cm}|}
\hline
\textbf{Column} & \textbf{Type} & \textbf{Constraints} & \textbf{Description} \\
\hline
\endhead
id & int(11) & PRIMARY KEY, AUTO\_INCREMENT & Unique subject identifier \\
\hline
name & varchar(100) & NOT NULL & Subject name \\
\hline
code & varchar(20) & UNIQUE & Subject code \\
\hline
teacher\_id & int(11) & FOREIGN KEY (user.id) & Assigned teacher \\
\hline
\end{longtable}

\subsubsection{GRADE Table}
\begin{longtable}{|p{3cm}|p{2cm}|p{2cm}|p{6cm}|}
\hline
\textbf{Column} & \textbf{Type} & \textbf{Constraints} & \textbf{Description} \\
\hline
\endhead
id & int(11) & PRIMARY KEY, AUTO\_INCREMENT & Unique grade identifier \\
\hline
value & float & NOT NULL & Grade value obtained \\
\hline
max\_value & float & DEFAULT 20.0 & Maximum possible grade \\
\hline
date & datetime & DEFAULT CURRENT\_TIMESTAMP & Grade assignment date \\
\hline
student\_id & int(11) & FOREIGN KEY (user.id) & Student who received grade \\
\hline
subject\_id & int(11) & FOREIGN KEY (subject.id) & Subject for the grade \\
\hline
teacher\_id & int(11) & FOREIGN KEY (user.id) & Teacher who assigned grade \\
\hline
\end{longtable}

\subsubsection{NOTE Table}
\begin{longtable}{|p{3cm}|p{2cm}|p{2cm}|p{6cm}|}
\hline
\textbf{Column} & \textbf{Type} & \textbf{Constraints} & \textbf{Description} \\
\hline
\endhead
id & int(11) & PRIMARY KEY, AUTO\_INCREMENT & Unique note identifier \\
\hline
data & varchar(1000) & NOT NULL & Note content \\
\hline
date & datetime & DEFAULT CURRENT\_TIMESTAMP & Note creation date \\
\hline
user\_id & int(11) & FOREIGN KEY (user.id) & Note author \\
\hline
\end{longtable}

\subsection{Database Relationships}

\begin{itemize}
    \item \textbf{One-to-Many:} User → Subjects (Teacher teaches multiple subjects)
    \item \textbf{One-to-Many:} User → Grades (Student receives multiple grades)
    \item \textbf{One-to-Many:} User → Grades (Teacher assigns multiple grades)
    \item \textbf{One-to-Many:} Subject → Grades (Subject has multiple grades)
    \item \textbf{One-to-Many:} User → Notes (User writes multiple notes)
\end{itemize}

\subsection{Sample Data Analysis}

\subsubsection{Current Users}
\begin{itemize}
    \item \textbf{Houssam} (ID: 2) - Student
    \item \textbf{Madara} (ID: 3) - Teacher
    \item \textbf{Kakashi} (ID: 4) - Teacher
\end{itemize}

\subsubsection{Current Subjects}
\begin{itemize}
    \item \textbf{Math} (Code: 101) - Taught by Madara
    \item \textbf{Phil} (Code: 102) - Taught by Kakashi
\end{itemize}

\subsubsection{Current Grades}
\begin{itemize}
    \item Houssam: 19/20 in Math (assigned by Madara)
    \item Houssam: 3/20 in Philosophy (assigned by Kakashi)
\end{itemize}

\section{Website Functionality Demonstration}

This section provides a comprehensive walkthrough of the implemented web application, demonstrating each feature with detailed explanations and visual evidence of functionality.

\subsection{User Authentication System}

\subsubsection{User Registration Process}
The registration system allows new users to create accounts with role-based access control.

\begin{figure}[h]
\centering
\includegraphics[width=0.8\textwidth]{registration_page.png}
\caption{Registration Interface - New User Account Creation}
\label{fig:registration}
\end{figure}

\textbf{Key Features Demonstrated:}
\begin{itemize}
    \item Role selection dropdown (Teacher/Student) - determines user permissions
    \item Email validation ensures unique user identification
    \item Password security with client-side validation
    \item Form validation with error handling and user feedback
    \item Clean, responsive Bootstrap-based interface design
\end{itemize}

\subsubsection{User Login Authentication}
The login system provides secure access with session management.

\begin{figure}[h]
\centering
\includegraphics[width=0.8\textwidth]{login_page.png}
\caption{Login Interface - Secure User Authentication}
\label{fig:login}
\end{figure}

\textbf{Authentication Features:}
\begin{itemize}
    \item Email-based user identification
    \item Password verification using Werkzeug hashing
    \item Flask-Login session management for persistent login
    \item Automatic role-based redirection after successful login
    \item Error messaging for invalid credentials
\end{itemize}

\subsection{Teacher Dashboard and Functionality}

\subsubsection{Teacher Main Dashboard}
The teacher dashboard serves as the central hub for all teaching-related activities.

\begin{figure}[h]
\centering
\includegraphics[width=0.8\textwidth]{teacher_dashboard.png}
\caption{Teacher Dashboard - Central Management Interface}
\label{fig:teacher_dash}
\end{figure}

\textbf{Dashboard Components:}
\begin{itemize}
    \item Welcome message with personalized greeting using user's first name
    \item Navigation menu with role-specific options
    \item Quick access buttons to primary functions (Create Subject, Manage Grades)
    \item Subject overview showing all subjects assigned to the teacher
    \item Real-time data display from the MySQL database
\end{itemize}

\subsubsection{Subject Creation Interface}
Teachers can create and manage academic subjects through this interface.

\begin{figure}[h]
\centering
\includegraphics[width=0.8\textwidth]{create_subject.png}
\caption{Subject Creation - Adding New Academic Courses}
\label{fig:create_subject}
\end{figure}

\textbf{Subject Management Features:}
\begin{itemize}
    \item Subject name input with validation
    \item Unique subject code assignment for identification
    \item Automatic teacher assignment (current logged-in teacher)
    \item Database integration with foreign key relationships
    \item Form validation preventing duplicate subject codes
\end{itemize}

\subsubsection{Grade Assignment System}
The grade management system allows teachers to assign and track student performance.

\begin{figure}[h]
\centering
\includegraphics[width=0.8\textwidth]{add_grade.png}
\caption{Grade Assignment - Student Performance Tracking}
\label{fig:add_grade}
\end{figure}

\textbf{Grade Assignment Process:}
\begin{itemize}
    \item Student selection from enrolled users with 'student' role
    \item Subject selection from teacher's assigned subjects
    \item Grade value input with maximum value specification (default: 20)
    \item Automatic timestamp recording for grade assignment
    \item Percentage calculation for performance analysis
    \item Database relationships maintaining data integrity
\end{itemize}

\subsubsection{Grade Management Interface}
Teachers can view, edit, and delete grades through this comprehensive management interface.

\begin{figure}[h]
\centering
\includegraphics[width=0.8\textwidth]{manage_grades.png}
\caption{Grade Management - Edit and Delete Operations}
\label{fig:manage_grades}
\end{figure}

\textbf{Management Capabilities:}
\begin{itemize}
    \item Tabular display of all grades assigned by the teacher
    \item Student name, subject, grade value, and percentage display
    \item Edit functionality for grade corrections
    \item Delete functionality with confirmation dialogs
    \item Date tracking for audit purposes
    \item Real-time percentage calculations
\end{itemize}

\subsection{Student Dashboard and Grade Viewing}

\subsubsection{Student Grade Overview}
Students can view their academic performance through a personalized dashboard.

\begin{figure}[h]
\centering
\includegraphics[width=0.8\textwidth]{student_dashboard.png}
\caption{Student Dashboard - Personal Academic Performance}
\label{fig:student_dash}
\end{figure}

\textbf{Student View Features:}
\begin{itemize}
    \item Personalized welcome message with student's name
    \item Complete grade history display
    \item Subject-wise grade organization
    \item Teacher identification for each grade
    \item Percentage calculations for performance analysis
    \item Date information for grade assignment tracking
    \item Read-only access ensuring data security
\end{itemize}

\subsection{Navigation and User Experience}

\subsubsection{Role-Based Navigation System}
The navigation system adapts based on user roles and authentication status.

\begin{figure}[h]
\centering
\includegraphics[width=0.8\textwidth]{navigation_bar.png}
\caption{Navigation System - Role-Aware Interface}
\label{fig:navigation}
\end{figure}

\textbf{Navigation Features:}
\begin{itemize}
    \item Dynamic role indicator showing current user's role (Teacher/Student)
    \item Conditional menu items based on user permissions
    \item Logout functionality with session termination
    \item Responsive design for mobile compatibility
    \item Consistent branding and visual hierarchy
    \item Bootstrap integration for professional appearance
\end{itemize}

\subsection{Technical Implementation Highlights}

\textbf{Security Measures:}
\begin{itemize}
    \item Password hashing using Werkzeug security functions
    \item Session-based authentication with Flask-Login
    \item Role-based access control preventing unauthorized access
    \item CSRF protection through Flask's built-in security features
\end{itemize}

\textbf{Database Integration:}
\begin{itemize}
    \item Real-time data synchronization with MySQL database
    \item Foreign key relationships maintaining data integrity
    \item Automatic timestamp recording for audit trails
    \item Efficient query optimization for performance
\end{itemize}

\textbf{User Experience Design:}
\begin{itemize}
    \item Responsive Bootstrap framework for cross-device compatibility
    \item Intuitive navigation with clear visual hierarchy
    \item Flash messaging system for user feedback
    \item Confirmation dialogs for destructive operations
    \item Clean, professional interface design
\end{itemize}

\section{Technical Implementation}

\subsection{Flask Application Structure}

\begin{lstlisting}[caption=Application Factory Pattern]
def create_app():
    app = Flask(__name__)
    app.config['SECRET_KEY'] = 'createapp'
    app.config['SQLALCHEMY_DATABASE_URI'] = 'mysql+mysqlconnector://root:@localhost/user'
    
    db.init_app(app)
    
    from .views import views
    from .auth import auth
    app.register_blueprint(views, url_prefix='/')
    app.register_blueprint(auth, url_prefix='/')
    
    return app
\end{lstlisting}

\subsection{Database Models Implementation}

\begin{lstlisting}[caption=SQLAlchemy Models]
class User(db.Model, UserMixin):
    id = db.Column(db.Integer, primary_key=True)
    email = db.Column(db.String(150), unique=True)
    password = db.Column(db.String(150))
    first_name = db.Column(db.String(150))
    role = db.Column(db.String(20), default='student')
    
    # Relationships
    grades = db.relationship('Grade', foreign_keys='Grade.student_id')
    taught_grades = db.relationship('Grade', foreign_keys='Grade.teacher_id')
\end{lstlisting}

\section{Testing \& Verification}

\subsection{Database Testing}
\begin{itemize}
    \item ✓ Table creation and structure verification
    \item ✓ Foreign key constraints validation
    \item ✓ Data insertion and retrieval testing
    \item ✓ Relationship integrity verification
\end{itemize}

\subsection{Functionality Testing}
\begin{itemize}
    \item ✓ User registration and login workflows
    \item ✓ Role-based access control
    \item ✓ CRUD operations for grades and subjects
    \item ✓ Data validation and error handling
\end{itemize}

\section{Conclusion}

Step 2 has been successfully completed with all deliverables implemented and operational. The system now features:

\begin{itemize}
    \item Fully functional MySQL database with proper relationships
    \item Complete user authentication and authorization system
    \item Operational teacher and student dashboards
    \item Grade management system with full CRUD capabilities
    \item Responsive web interface with intuitive navigation
\end{itemize}

The database contains sample data demonstrating all relationships and the website functionalities are fully operational and ready for comprehensive testing.

\vspace{1cm}

\noindent\textbf{Repository URL:} \url{https://github.com/Dioxeur/flask-grade-management-system}

\noindent\textbf{Project Status:} Step 2 Completed Successfully

\section{Website Functionality Screenshots}

\subsection{Authentication System}

\subsubsection{Login Interface}
\begin{figure}[h]
\centering
\includegraphics[width=0.8\textwidth]{login_page.png}
\caption{User Login Page - Email and Password Authentication}
\label{fig:login}
\end{figure}

\subsubsection{Registration Interface}
\begin{figure}[h]
\centering
\includegraphics[width=0.8\textwidth]{registration_page.png}
\caption{User Registration Page - Role Selection (Teacher/Student)}
\label{fig:registration}
\end{figure}

\subsection{Teacher Dashboard}

\subsubsection{Teacher Main Dashboard}
\begin{figure}[h]
\centering
\includegraphics[width=0.8\textwidth]{teacher_dashboard.png}
\caption{Teacher Dashboard - Subject Management and Grade Operations}
\label{fig:teacher_dash}
\end{figure}

\subsubsection{Subject Creation}
\begin{figure}[h]
\centering
\includegraphics[width=0.8\textwidth]{create_subject.png}
\caption{Create Subject Page - Adding New Academic Subjects}
\label{fig:create_subject}
\end{figure}

\subsubsection{Grade Management}
\begin{figure}[h]
\centering
\includegraphics[width=0.8\textwidth]{add_grade.png}
\caption{Add Grade Interface - Assigning Grades to Students}
\label{fig:add_grade}
\end{figure}

\begin{figure}[h]
\centering
\includegraphics[width=0.8\textwidth]{manage_grades.png}
\caption{Manage Grades Page - Edit and Delete Grade Operations}
\label{fig:manage_grades}
\end{figure}

\subsection{Student Dashboard}

\subsubsection{Student Grade View}
\begin{figure}[h]
\centering
\includegraphics[width=0.8\textwidth]{student_dashboard.png}
\caption{Student Dashboard - Personal Grade Overview with Percentages}
\label{fig:student_dash}
\end{figure}

\subsection{Navigation System}

\subsubsection{Role-Based Navigation}
\begin{figure}[h]
\centering
\includegraphics[width=0.8\textwidth]{navigation_bar.png}
\caption{Navigation Bar - Role Indicator and Menu Options}
\label{fig:navigation}
\end{figure}

\end{document}


