\documentclass[12pt,a4paper]{article}
\usepackage[utf8]{inputenc}
\usepackage[english]{babel}
\usepackage{geometry}
\usepackage{fancyhdr}
\usepackage{graphicx}
\usepackage{hyperref}
\usepackage{listings}
\usepackage{xcolor}
\usepackage{booktabs}
\usepackage{longtable}
\usepackage{array}

% Page setup
\geometry{margin=1in}
\pagestyle{fancy}
\fancyhf{}
\rhead{Step 1 Report}
\lhead{Flask Grade Management System}
\cfoot{\thepage}

% Code listing setup
\lstset{
    basicstyle=\ttfamily\small,
    backgroundcolor=\color{gray!10},
    frame=single,
    breaklines=true,
    showstringspaces=false
}

% Hyperlink setup
\hypersetup{
    colorlinks=true,
    linkcolor=blue,
    urlcolor=blue,
    citecolor=blue
}

\title{\textbf{Step 1 Report: Prep Work} \\ 
       \large Tools Installation, Environment Setup \& Git Repository \\
       \vspace{0.5cm}
       \large Flask Grade Management System}
\author{Gabriel Mokhele (Dioxeur)}
\date{\today}

\begin{document}

\maketitle
\thispagestyle{empty}

\newpage

\tableofcontents
\newpage

\section{Executive Summary}

This report documents the successful completion of Step 1 deliverables for the Flask Grade Management System project. The primary objectives included establishing a robust development environment, implementing version control systems, and preparing the foundation for subsequent development phases.

\subsection{Project Overview}
\begin{itemize}
    \item \textbf{Project Name:} Flask Grade Management System
    \item \textbf{Developer:} Gabriel Mokhele (GitHub: Dioxeur)
    \item \textbf{Repository:} \url{https://github.com/Dioxeur/flask-grade-management-system}
    \item \textbf{Status:} \textcolor{green}{\textbf{COMPLETED}}
    \item \textbf{Completion Date:} \today
\end{itemize}

\section{Deliverables Status}

\begin{longtable}{|p{2cm}|p{4cm}|p{2cm}|p{6cm}|}
\hline
\textbf{Step} & \textbf{Deliverable} & \textbf{Due Date} & \textbf{Format} \\
\hline
\endhead
1 & Prep work & Day 1 & Tools installation, setting up the environment, Git deposit \\
\hline
\end{longtable}

\section{Technical Implementation}

\subsection{Tools Installation}

\subsubsection{Python Environment}
\begin{itemize}
    \item \textbf{Python Version:} 3.7+ (Verified operational)
    \item \textbf{Package Manager:} pip (Latest version)
    \item \textbf{Virtual Environment Tool:} venv module
\end{itemize}

\subsubsection{Database System}
\begin{itemize}
    \item \textbf{Database Engine:} MySQL
    \item \textbf{Management Interface:} phpMyAdmin via XAMPP
    \item \textbf{Database Name:} \texttt{user}
    \item \textbf{Connection Status:} Verified with Flask-SQLAlchemy
\end{itemize}

\subsubsection{Development Tools}
\begin{itemize}
    \item \textbf{Version Control:} Git
    \item \textbf{Remote Repository:} GitHub
    \item \textbf{Terminal:} Command line interface configured
\end{itemize}

\subsection{Environment Setup}

\subsubsection{Virtual Environment Configuration}
The virtual environment was successfully created and activated:

\begin{lstlisting}[language=bash, caption=Virtual Environment Setup]
# Environment creation
python -m venv .venv

# Activation (Windows)
.venv\Scripts\activate

# Verification
pip list
\end{lstlisting}

\subsubsection{Dependencies Management}
A comprehensive \texttt{requirements.txt} file was generated containing all necessary packages:

\begin{lstlisting}[caption=Key Dependencies]
Flask==2.3.3
Flask-Login==0.6.3
Flask-SQLAlchemy==3.0.5
mysql-connector-python==8.1.0
Werkzeug==2.3.7
\end{lstlisting}

\subsubsection{Project Structure}
The following professional project structure was established:

\begin{lstlisting}[caption=Project Directory Structure]
flask-grade-management-system/
├── .venv/                    # Virtual environment (ignored)
├── website/                  # Flask application package
│   ├── __init__.py          # Application factory
│   ├── auth.py              # Authentication routes
│   ├── views.py             # Main application routes
│   ├── models.py            # Database models
│   └── templates/           # HTML templates
│       ├── base.html
│       ├── login.html
│       ├── sign_up.html
│       ├── home.html
│       ├── teacher_dashboard.html
│       ├── student_dashboard.html
│       ├── create_subject.html
│       ├── add_grade.html
│       ├── edit_grade.html
│       └── manage_grades.html
├── main.py                  # Application entry point
├── requirements.txt         # Dependencies specification
├── .gitignore              # Git ignore configuration
└── README.md               # Project documentation
\end{lstlisting}

\subsection{Git Repository Implementation}

\subsubsection{Local Repository Initialization}
\begin{lstlisting}[language=bash, caption=Git Initialization]
# Repository initialization
git init

# Initial file staging
git add .

# Initial commit
git commit -m "Initial commit: Flask Grade Management System"
\end{lstlisting}

\subsubsection{Version Control Configuration}
A comprehensive \texttt{.gitignore} file was implemented to exclude:

\begin{lstlisting}[caption=Git Ignore Configuration]
# Python
__pycache__/
*.pyc
*.pyo

# Virtual Environment
.venv/

# Database files
*.sql
*.db

# Environment variables
.env

# IDE files
.vscode/

# OS files
.DS_Store
Thumbs.db
\end{lstlisting}

\subsubsection{Remote Repository Setup}
\begin{lstlisting}[language=bash, caption=GitHub Integration]
# Remote repository connection
git remote add origin https://github.com/Dioxeur/flask-grade-management-system.git

# Branch configuration
git branch -M main

# Initial push
git push -u origin main
\end{lstlisting}

\section{Application Architecture}

\subsection{Flask Application Design}
The application follows Flask best practices with a modular structure:

\begin{itemize}
    \item \textbf{Application Factory Pattern:} Implemented in \texttt{website/\_\_init\_\_.py}
    \item \textbf{Blueprint Architecture:} Separate modules for authentication and views
    \item \textbf{Database Integration:} SQLAlchemy ORM with MySQL backend
    \item \textbf{User Authentication:} Flask-Login for session management
\end{itemize}

\subsection{Database Models}
Three primary models were established:
\begin{itemize}
    \item \textbf{User Model:} Handles teacher and student accounts
    \item \textbf{Subject Model:} Manages academic subjects
    \item \textbf{Grade Model:} Tracks student grades and assignments
\end{itemize}

\section{Quality Assurance}

\subsection{Code Quality Standards}
\begin{itemize}
    \item ✓ Clean project structure following Flask conventions
    \item ✓ Proper separation of concerns (MVC pattern)
    \item ✓ Comprehensive error handling implementation
    \item ✓ Security best practices (password hashing, session management)
\end{itemize}

\subsection{Documentation Standards}
\begin{itemize}
    \item ✓ Detailed README with installation instructions
    \item ✓ Clear project description and feature specifications
    \item ✓ Professional repository presentation
    \item ✓ Inline code documentation
\end{itemize}

\subsection{Version Control Best Practices}
\begin{itemize}
    \item ✓ Meaningful commit messages
    \item ✓ Proper file organization and .gitignore configuration
    \item ✓ Remote backup and collaboration setup
    \item ✓ Branch protection and workflow preparation
\end{itemize}

\section{Challenges and Solutions}

\subsection{Cross-Platform Compatibility}
\textbf{Challenge:} Git line ending warnings (LF/CRLF conversion) on Windows system.

\textbf{Solution:} Configured Git to handle cross-platform line ending conversion automatically, ensuring compatibility across different operating systems.

\subsection{Remote Repository Configuration}
\textbf{Challenge:} "Remote origin already exists" error during GitHub connection.

\textbf{Solution:} Properly removed existing remote configuration and re-established clean connection to GitHub repository.

\section{Testing and Verification}

\subsection{Environment Testing}
\begin{itemize}
    \item ✓ Virtual environment activation and deactivation
    \item ✓ Package installation and dependency resolution
    \item ✓ Database connection and basic operations
    \item ✓ Flask application startup and basic routing
\end{itemize}

\subsection{Repository Verification}
\begin{itemize}
    \item ✓ Local Git operations (add, commit, status)
    \item ✓ Remote push and pull operations
    \item ✓ GitHub repository accessibility and presentation
    \item ✓ File ignore functionality verification
\end{itemize}

\section{Conclusion}

Step 1 has been successfully completed with all deliverables met according to the specified timeline. The project now has a solid foundation featuring:

\begin{itemize}
    \item Professional development environment with proper tool configuration
    \item Robust version control system with remote backup
    \item Clear documentation standards and project structure
    \item Scalable application architecture following industry best practices
\end{itemize}

All tools have been installed and configured, the development environment is operational, and the Git repository has been successfully established with remote backup on GitHub.

\vspace{1cm}

\noindent\textbf{Repository URL:} \url{https://github.com/Dioxeur/flask-grade-management-system}

\noindent\textbf{Project Status:} Step 1 Completed Successfully

\end{document}
